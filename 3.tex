\documentclass{article}
\usepackage{amsthm}
\usepackage{amsmath}
\usepackage{amssymb}
\usepackage{lmodern}
\usepackage{hyperref}
\usepackage{cleveref}
\usepackage{mathtools}
\usepackage{stmaryrd}

\setlength{\parindent}{0pt}
\linespread{1.4}

\theoremstyle{definition}
\newtheorem{theorem}{Theorem}
\newtheorem{proposition}{Proposition}
\newtheorem{lemma}{Lemma}
\newtheorem{example}{Example}
\newtheorem{definition}{Definition}
\newtheorem{remark}{Remark}
\newtheorem{property}{Property}
\newtheorem{exercise}{Exercise}

\providecommand*{\definitionautorefname}{Definition}
\providecommand*{\theoremautorefname}{Theorem}
\providecommand*{\lemmaautorefname}{Lemma}
\providecommand*{\propositionautorefname}{Proposition}
\providecommand*{\definitionautorefname}{Definition}
\providecommand*{\exampleautorefname}{Example}
\providecommand*{\remarkautorefname}{Remark}
\providecommand*{\propertyautorefname}{Property}
\providecommand*{\exerciseautorefname}{Exercise}

\begin{document}
\begin{example}
    If $a>1$, then $\lim\limits_{n \to \infty} \frac{1}{a^n}=0$.
\end{example}

\begin{proof}[Solution]
$\frac{1}{a^n}=\frac{1}{(1+(a-1))^n}\leq\frac{1}{1+n(a-1)}$
\end{proof}

\begin{exercise}[Squeeze theorem]
    If $n\in \mathbb{N}$, $a_n\leq c_n\leq b_n$.

    $$\lim_{n\to \infty} a_n=\lim_{n\to \infty} b_n = L \implies \lim_{n\to \infty} c_n = L$$
\end{exercise}

\begin{theorem}
    If $a_n$ is increasing and $\{a_n \mid n\in \mathbb{N}\}$ has an upper bound, then $a_n$ converges.
\end{theorem}
\begin{proof}
    $\{a_n\mid n\in \mathbb{N}\}$ has an upper bound $\implies \sup\{a_n\mid n\in \mathbb{N}\}$ exists.

    For convenience, we denote $\sup\{a_n\mid n\in \mathbb{N}\}$ by $L$.

    $\forall \varepsilon>0, L-\varepsilon<L$, and hence $\exists N\in \mathbb{N}\ \llbracket L-\varepsilon<a_N\rrbracket$($\implies L-\varepsilon$ is not an upper bound.)
    

    $\forall n \geq \mathbb{N}\ \llbracket L-\varepsilon<a_N\leq a_n\leq L < L + \varepsilon\rrbracket \implies |a_n-L|<\varepsilon$
\end{proof}

\begin{definition}
    A sequence of intervals $I_n\ (n\in \mathbb{N})$ is nested if $I_n\neq \varnothing$ and $I_{n+1} \subseteq I_n$ for all n $\in \mathbb{N}$.
\end{definition}

\begin{theorem}[Nested intervals theorem]
    If $I_n\ (n\in \mathbb{N})$ is a sequence of bounded closed nested intervals, then $$\bigcap_{n\in \mathbb{N}} I_n \neq \varnothing$$
\end{theorem}

\begin{proof}

    Write $I_n=[a_n,b_n]$ $(n\in \mathbb{N})$

    $I_n$ is nested $\iff a_n\leq b_n,\,a_n \nearrow$ and $b_n \searrow$

    $\forall n, m\in \mathbb{N}\quad a_n\leq a_{\max(n, m)}\leq b_{\max(n, m)}\leq b_m$
    
    In other words, for every $m\in \mathbb{N}$, $b_m$ is an upper bound of $\{a_n \mid n\in \mathbb{N}\}$.

    Let $c=\lim_{n\to \infty} a_n$. Then $a_n\leq c\leq b_mm$ for all $m\in \mathbb{N} \implies$ $$c \in \bigcap_{n\in \mathbb{N}} I_n$$
\end{proof}

\begin{exercise}
    What if 

    \begin{enumerate}
        \item $I_n=(a_n, b_n)$, nested, but $a_n\nearrow\nearrow$ and $b_n\searrow\searrow$.
        \item $I_n=(a_n, \infty)$, nested and $\{a_n\mid n\in \mathbb{N}\}$ is bounded from above.
    \end{enumerate}
\end{exercise}

\begin{exercise}
    Prove the \textbf{Dedekind gapless property} using \textbf{Archimedean property} and the \textbf{nested intervals theorem}.
\end{exercise}

\begin{definition}
    A sequence $a_n\ (n\in \mathbb{N})$ in $\mathbb{R}$ is a Cauchy sequence if 
    $$\forall \varepsilon > 0 \quad \exists N\in \mathbb{N}\ \llbracket n, m\geq\mathbb{N}\implies|a_n-a_m|<\varepsilon\rrbracket$$
\end{definition}

Obviously, 
\begin{itemize}
    \item $a_n$ is convergent$\implies a_n$ is Cauchy sequence
    \item $a_n$ is Cauchy sequence$\implies a_n$ is bounded
\end{itemize}

\begin{definition}
    Let $a_n\ (n\in \mathbb{N})$ be a \emph{bounded} sequence in $\mathbb{R}$.

    $u_n:=\sup\{a_m\mid m \geq n\}$, $l_n:=\inf\{a_m\mid m \geq n\}$
    
    $\forall n\in \mathbb{N}\ \llbracket l_n\leq a_n \leq u_n\rrbracket$

    $\mathop{\overline{\lim}}\limits_{n\to \infty} a_n:=\lim\limits_{n\to \infty} u_n$, $\mathop{\underline{\lim}}\limits_{n\to \infty} a_n:=\lim\limits_{n\to \infty} l_n$
\end{definition}

\begin{exercise}
    $a_n$ converges $\iff \mathop{\overline{\lim}}\limits_{n\to \infty} a_n = \mathop{\underline{\lim}}\limits_{n\to \infty} a_n$, and if any of both sides holds, then $\lim\limits_{n\to \infty} a_n = \mathop{\overline{\lim}}\limits_{n\to \infty} a_n = \mathop{\underline{\lim}}\limits_{n\to \infty} a_n$.
\end{exercise}

\begin{theorem}
    Let $a_n\ (n\in \mathbb{N})$ be a sequence in $\mathbb{R}$.
    $$a_n\text{ is convergent }\iff a_n\text{ is a Cauchy sequence}$$
\end{theorem}
\end{document}
