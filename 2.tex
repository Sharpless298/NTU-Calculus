\documentclass{article}
\usepackage{amsthm}
\usepackage{amsmath}
\usepackage{amssymb}
\usepackage{lmodern}
\usepackage{hyperref}
\usepackage{cleveref}
\usepackage{mathtools}

\setlength{\parindent}{0pt}

\theoremstyle{definition}
\newtheorem{theorem}{Theorem}
\newtheorem{proposition}{Proposition}
\newtheorem{lemma}{Lemma}
\newtheorem{example}{Example}
\newtheorem{definition}{Definition}
\newtheorem{remark}{Remark}
\newtheorem{property}{Property}
\newtheorem{exercise}{Exercise}

\providecommand*{\definitionautorefname}{Definition}
\providecommand*{\theoremautorefname}{Theorem}
\providecommand*{\lemmaautorefname}{Lemma}
\providecommand*{\propositionautorefname}{Proposition}
\providecommand*{\definitionautorefname}{Definition}
\providecommand*{\exampleautorefname}{Example}
\providecommand*{\remarkautorefname}{Remark}
\providecommand*{\propertyautorefname}{Property}
\providecommand*{\exerciseautorefname}{Exercise}

\begin{document}

\begin{definition}[Supremum / Least Upper Bound]
Let $S \subset \mathbb{R}$ be a non-empty set that is bounded above.  
A real number $u$ is called the \emph{supremum} (least upper bound) of $S$, denoted by $\sup S$, if
\begin{enumerate}
  \item $u$ is an upper bound of $S$, i.e., $s \le u$ for all $s \in S$;
  \item for any $\epsilon > 0$, there exists $s \in S$ such that $s > u - \epsilon$.
\end{enumerate}
\end{definition}

\begin{definition}[Infimum / Greatest Lower Bound]
Let $S \subset \mathbb{R}$ be a non-empty set that is bounded below.  
A real number $l$ is called the \emph{infimum} (greatest lower bound) of $S$, denoted by $\inf S$, if
\begin{enumerate}
  \item $l$ is a lower bound of $S$, i.e., $s \ge l$ for all $s \in S$;
  \item for any $\epsilon > 0$, there exists $s \in S$ such that $s < l + \epsilon$.
\end{enumerate}
\end{definition}

By convention,
\begin{enumerate}
    \item $\sup S = \infty\ (\text{resp.\ } \inf S = -\infty)$
whenever $S$ has no upper (resp.\ lower) bound.
    \item $S$ is bounded above (resp.\ below) $\iff S$ has a upper (resp.\ lower) bound.
\end{enumerate}

\begin{remark}
Every $r \in \mathbb{R}$ is a upper bound and lower bound of $\varnothing$.
\end{remark}

\begin{definition}[Dedekind cut]
Let $A, B \subseteq \mathbb{R}$. We say that $(A, B)$ is a \emph{Dedekind cut} of $\mathbb{R}$ if following conditions hold:
\begin{enumerate}
    \item $A \neq \varnothing \neq B$;
    \item $A \cup B = \mathbb{R}$;
    \item $\forall a \in A, \forall b \in B, \ a < b$.
\end{enumerate}
\end{definition}

\begin{property}[Dedekind's gapless property]\label{p1}
    If (A,B) is a Dedekind cut, then exactly one of the following happens:
    \begin{enumerate}
        \item $\max A$ exists but $\min B$ doesn't;
        \item $\min B$ exists but $\max A$ doesn't.
    \end{enumerate}
\end{property}

\begin{exercise}
    We may define Dedekind cuts of $\mathbb{Q}$ and $\mathbb{Z}$ similarly.
    Does the $\textbf{\autoref{p1}}$ hold for $\mathbb{Q}$ or $\mathbb{Z}$?
\end{exercise}

\begin{property}[Least upper bound property]
    If $S$ has a upper bound, then $\sup S$ exists.
\end{property}

\begin{proof}
    Let $\varnothing \neq S \subseteq \mathbb{R}$, $B\coloneqq\{b\in\mathbb{R}\mid b \text{ is upper bound of } S\}$ and $A\coloneqq\mathbb{R} \setminus B.$
    We need to show that $\min B$ exists. First, we prove that $(A,B)$ forms a \emph{Dedekind cut} of $\mathbb{R}$.
    \begin{enumerate}
        \item $S \neq \varnothing \implies A \neq \varnothing$, and $S$ has an upper bound $\iff B  \neq \varnothing$
        \item $A=\mathbb{R}\setminus B \implies A\cup B = \mathbb{R}$
        \item For $a\in A \text{ and } b\in B$, we need to show that $a<b$. 
            Assume the contrary, that $a\geq b$. Hence $a$ is an upper bound of $S$, i.e., $a\in B$ and
            $a\in A\cap B = \varnothing$. Therefore, our assumption is false, and $a < b$.
    \end{enumerate}
    Hence $(A,B)$ is a \emph{Dedekind cut} of $\mathbb{R}$.

    Next we prove that $\min B$ exists.
    Again, we assume the contrary, that $\max A$ exists, denoted by $a_0$.
    \begin{gather*}
        a_0\in A \iff a_0 \notin B \iff a_0 \text{ is not an upper bound of } S \\
        \iff \exists s_0 \in S \text{ such that } a_0<s_0
    \end{gather*}
    Choose $x$ such that $a_0<x<s_0$. Then $a_0<x \iff x\in B$, hence $x$ is an upper bound of $S$, but $x < s_0$.
    Therefore, our assumption is false, and $\min B$ exists.(\autoref{p1})
\end{proof}

\begin{property}[Greatest lower bound property]
    If $S$ has a lower bound, then $\inf S$ exists.
\end{property}

\begin{exercise}[Archimedean property] 
    Prove that
    $$\forall r \in \mathbb{R}, r>0 \implies \exists n \in \mathbb{N} \text{ such that } \frac{1}{n} < r$$
\end{exercise}

\textbf{Hint:} Rephrase this statement in terms of the set $S = \mathbb{N} \subseteq \mathbb{R}$ 
and consider its upper bounds.

\begin{definition}[Limit of a Sequence]
Let $(a_n)$ be a sequence. We say that $(a_n)$ \emph{converges} to a real number $L$, or that $L$ is the \emph{limit} of $(a_n)$, if $\forall \varepsilon > 0$, there exists a positive integer $N$ such that $\forall n > N$,
\[
|a_n - L| < \varepsilon.
\]
In this case, we write
\[
\lim_{n \to \infty} a_n = L.
\]
\end{definition}

\begin{exercise}\label{e3}
    Prove that
    \begin{enumerate}
        \item $\lim\limits_{n \to \infty} a_n = L, \lim\limits_{n \to \infty} a_n = M \implies L=M$
        \item $a_n\ (n\in \mathbb{N})$ is convergent $\implies \{a_n\mid n \in \mathbb{N}\}$ is bounded.
        \item If $a_n\leq b_n$ for all $n\in \mathbb{N}$, $\lim\limits_{n\to \infty} a_n =L$ and $\lim\limits_{n\to \infty} b_n=M$, then $L\leq M$.
            What if "$\leq$" is replaced by "$<$"?
    \end{enumerate}
\end{exercise}

\begin{remark}
    Modifying or removing \emph{finitely many terms} of a sequence $(a_n)$ does not affect its convergence or divergence, nor the value of its limit if it exists.
\end{remark}

\begin{proposition}\label{prop:1}
    If $\lim\limits_{n\to \infty}a_n=L$ and $\lim\limits_{n\to \infty}b_n=M$, then
    \begin{enumerate}
        \item $\lim\limits_{n\to \infty}(a_n\pm b_n)=L\pm M$;
        \item $\lim\limits_{n\to \infty}a_n b_n=L\cdot M$;
        \item If $M\neq 0$, then $b_n \neq 0$ for all but finitely many $n$, and $\lim\limits_{n\to \infty}\frac{a_n}{b_n}=\frac{L}{M}$.
    \end{enumerate}
\end{proposition}

$Proof.$
\newline
1.
Consider $\lvert (a_n\pm b_n)-(L\pm M)\rvert$.
\begin{displaymath}
\lvert(a_n\pm b_n)-(L\pm M)\rvert=\lvert(a_n-L)\pm (b_n-M)\leq \lvert a_n-L\rvert+\lvert b_n-M\rvert
\end{displaymath}
\begin{gather*}
    \forall \varepsilon>0,\,\exists N_1,N_2\in \mathbb{N}[n\geq N_1 \implies \lvert a_n-L\rvert<\frac{\varepsilon}{2}]\\
\text{ and } [n\geq N_2 \implies \lvert b_n-M\rvert<\frac{\varepsilon}{2}]
\end{gather*}
Let $N=\max\{N_1,N_2\}$. Then $$n\geq N\implies \lvert(a_n\pm b_n)-(L\pm M)\rvert \leq \lvert a_n-L\rvert+\lvert b_n-M\rvert<\frac{\varepsilon}{2}+\frac{\varepsilon}{2}=\varepsilon$$
\qed

2.
Consider $\lvert a_n b_n-LM\rvert$.

\begin{gather*}
    |a_n b_n - LM|
    = |a_n b_n - L b_n + L b_n - LM| \\
    \leq |a_n - L| \, |b_n| + |L| \, |b_n - M|.
\end{gather*}
Since $b_n \to M$, the sequence $(b_n)$ is bounded (By $\textbf{\autoref{e3}}$).
Thus, there exists $c > 0$ such that $|b_n| \leq c$ and $|L| \leq c$ for all $n\in \mathbb{N}$. 
Therefore,
\[
    |a_n b_n - LM| \leq c |a_n - L| + c\, |b_n - M|.
\]
\[
    \forall \varepsilon>0,\,\exists N \in \mathbb{N}[n\geq N\implies |a_n-L|<\frac{\varepsilon}{2c}\text{ and }|b_n-M|<\frac{\varepsilon}{2c}]
\]
\[
    \implies |a_n b_n-LM|<c\frac{\varepsilon}{2c}+c\frac{\varepsilon}{2c}=\varepsilon
\]
\qed

\begin{exercise}
    Prove 3. in $\textbf{\autoref{prop:1}}$.
\end{exercise}
\end{document}
